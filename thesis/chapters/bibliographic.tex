\chapter{Hinweise zur Literatur}
\label{sec:references}

\section{Literatursuche}

Der Vollzugang zu einigen Publikationen ist nur intern aus dem TU-Netz möglich. Um auf möglichst viele Papers extern zugreifen zu können, wird von der TU Wien eine \href{http://www.zid.tuwien.ac.at/tunet/vpn/extern/}{VPN-Zugangsmöglichkeit} angeboten, diesen VPN-Zugang bitte gleich einrichten.

Besonders ergiebig sind folgende Search-Engines:\\
\href{http://academic.research.microsoft.com/}{Microsoft Academic}\\
\href{http://dl.acm.org/}{ACM-Datenbank}\\
\href{http://scholar.google.com/}{Google Scholar}

Wir empfehlen, vor Beginn Ihrer Arbeit einige Diplomarbeiten, die am INSO oder generell an der Fakultät für Informatik verfaßt wurden, zu Ihrem Themenbereich zu suchen und Aufbau, Schreibstil, Art der Abbildungen etc. durchzuschauen. Arbeiten finden Sie \href{http://media.obvsg.at/tuw?query=grechenig&metaname=swishdefault&submit=Suche+starten&sbm=tuw*&lbm=*&lbc=*&searchtype=sim&.cgifields=metaname}{hier}.

Weitere Datenbanken und Suchmaschinen:\\
\href{http://rzblx1.uni-regensburg.de/ezeit/search.phtml?bibid=UBTUW&colors=7&lang=de}{Elektronische Zeitschriftenbibliothek der TU Wien}\\
\href{http://citeseer.ist.psu.edu/index;jsessionid=BF9BD5A89D42210F60E5CA88B40BAD9C}{Scientific Literature Digital Library (CiteSeer)}\\
\href{http://www.ingentaconnect.com/}{Ingenta}\\
\href{http://www.theiet.org/resources/inspec/}{INSPEC}

Journals:\\
\href{http://ieeexplore.ieee.org/}{IEEE - Institute of Electrical and Electronics Engineers, Inc. - Library}\\
\href{http://www.springerlink.com/?MUD=MP}{Verlag Springer - Springer Link}\\
\href{http://www.elsevier.com/wps/find/homepage.cws_home}{Elsevier}

Bibliotheken und Online-Kataloge:\\
\href{http://search.obvsg.at/primo_library/libweb/action/search.do?vid=ACC}{Online-Kataloge des Österreichischen Bibliothekenverbundes}\\
\href{http://aleph.ub.tuwien.ac.at/}{Online-Katalog der TU Wien} (ALEPH)\\
\href{http://www.informatik.uni-trier.de/}{Digital Bibliography \& Library Project (DBLP) of University of Trier}\\
\href{http://liinwww.ira.uka.de/bibliography/}{The Collection of Computer Science Bibliographies}

\section{BibLatex}

Biblatex bietet verschiedene Möglichkeiten an, um Literatur zu referenzieren. Die beiden häufigsten Befehle sind \verb|\cite| und \verb|\citeauthor|.

Beispiele wie referenziert werden kann:\\
\citeauthor{fankhauser:2009:softwaretechnik-security} beschreiben in \cite{fankhauser:2009:softwaretechnik-security} \dots\\
In \cite{schanes:2011:voip-fuzzer} zeigen \citeauthor{schanes:2011:voip-fuzzer} wie \dots
Weitere Informationen können in \cite{oasis:2010:homepage} von \citeauthor{oasis:2010:homepage} entnommen werden.

Wir empfehlen JabRef, um die Literaturdatenbank zu verwalten.
