%%%%%%%%%%%%%%%%%%%%%%%%%%%%%%%%%%%%%%%%%%%%%%%%%%%%%%%%%%%%%%%%%%%%%%%%
\chapter{Introduction}
\label{sec:introduction}
%%%%%%%%%%%%%%%%%%%%%%%%%%%%%%%%%%%%%%%%%%%%%%%%%%%%%%%%%%%%%%%%%%%%%%%%

er Einleitung eine modifizierte Version des Exposés als Basis verwendet werden.

%=======================================================================
\section{Problem description}
%=======================================================================

Wrist injuries are a major concern in today´s workplace environment.
Multiple billion USD are lost every year because of reduction in productivity, employee turnover and other indirect expenses associated with wrist pain\cite{freitas2017serious}. 
The most common type of wrist injuries are carpal tunnel syndrome, tendonitis and tenosynovitis. 
These injuries are also known under the umbrella term \gls{RSI}. 
These injuries do not manifest themselves immediately but after performing repetitive tasks on a daily basis for a long period of time. 
Effective wrist injury prevention exercises are a well studied subject. 
While they are not very complicated to execute, positive training results can only be achieved after regular application resulting in slow training progress. 
This slow progress often leads to a lack of motivation in patients, which in turn results in a high abandonment rate of exercise regimens\cite{campbell2001don}.

%=======================================================================
\section{Motivation}
%=======================================================================

A major goal of serious games is increasing motivation for tasks which are inherently dull which in turn makes these kind of games a perfect fit for the physiotherapy domain.
Applying serious games to therapeutic exercises has been a well researched topic over the last view years\cite{lohse2014virtual}\cite{wouters2013meta}\cite{gorvsivc2017competitive}. 
Most of these serious games choose a “mini-game” design pattern for translating therapeutic exercises into game-play. 
In the context of this exposé  a “mini-game” design pattern describes the approach of developing a distinct game mode for every single exercise. 
Since a single game mode only requires the execution of a single exercise there is no inherent modulation of movement patterns. 
Furthermore it cedes power of choice to the user, who might be overwhelmed by a plethora of options, or worse, chooses only game modes(e.g. exercises) which are easy to accomplish. 
While following a “mini-games” design pattern drastically reduces the complexity of game systems and ensures that exercises are represented in game as closely as possible to their real world counter parts the draw backs are substantial in terms of usability and user guidance.

Another source of problems in the serious games domain can be the over reliance on non-standard or hard to require hardware\cite{batista2019farmyo}.
An example of such hardware would be the Nintendo Wii game console which has often been used in serious games studies because of its intuitive control interface\cite{leder2008nintendo}. 
But production of the console has stopped in 2013 and with no alternative on the market, which would satisfy the same requirements, the prototypes developed for this system cannot be easily used in follow up experiments. 
Further more, being reliant on additional hardware increases the threshold for patients to use these serious games in their own homes\cite{wiemeyer2012serious}. 
This nullifies one of the greatest strengths for serious games. 
Usage without the need of expensive hardware, rehabilitation facilities and permanent professional supervision.

In summary, therapeutic serious games are currently suffering from major usability problems in hardware, software and design domains.
These drawbacks might be one of the reasons why therapeutic serious games are not really present in more casual settings, like at home or in the workplace.
Subsequently increasing usability and reducing entry barriers of use should be a major focus.


%=======================================================================
\section{Expected results}
%=======================================================================

The goal of this thesis is to design and develop a prototype for a therapeutic serious game in the domain of wrist injury prevention in conjunction with domain specific experts. 
All entry barriers for actually using the game, at home and in a professional therapeutic setting, should be reduced as much as possible. 
This means that the user does not need to install any additional software on any device required for running the game. 
Further more, the only hardware requirements should be devices which can reasonably be expected to be presented in an average household. 

\begin{enumerate}
	\item What are the functional requirements for a therapeutic serious game related to wrist injury prevention and is it possible to satisfy these requirements without following the “mini-game” design pattern approach? Furthermore, is therapeutic success still guaranteed if multiple therapeutic wrist exercises are merged into a single serious game?
	\item Is it possible to successfully develop a serious games prototype for therapeutic wrist exercises which facilitates the intuitive control schema introduced with specialized hardware, like the Nintendo Wii, without the need for said hardware? Are there any unwanted side effects introduced if the motion sensors of a smartphone are used as controller input? Last of all, how can these side effects, if present, be mitigated or even be totally removed?
	\item Is it technical feasible to develop a serious game with motion controls leveraging a distributed controller setup? In this exposé the term “distributed controller setup” is defined as a setup where the controller and the system displaying the game state are not connected directly with the likes of cables or wireless connections. All of the communication between the devices is handled by a web server. A main concern with this setup is the amount of latency introduced.
\end{enumerate}

%=======================================================================
\section{Structure of the thesis}
%=======================================================================



%%%%%%%%%%%%%%%%%%%%%%%%%%%%%%%%%%%%%%%%%%%%%%%%%%%%%%%%%%%%%%%%%%%%%%%%%%%%%%%%%%%%%%%%%%%%%%%%%%%%%%%%%%%%%%%%%%%%%%%%%%%%%%%%%%%%%%%%%%%%%%%%%%%%%%%%%%%%%%%%%%%%%%%%%%
% \section{General Information}
%
% This document is intended as a template and guideline and should support the author in the course of doing the master's thesis.
% Assessment criteria comprise the quality of the theoretical and/or practical work as well as structure, content and wording of the written master's thesis. Careful attention should be given to the basics of scientific work (e.g., correct citation).\footnote{Sample Footnote}
%
% \section{Organizational Issues}
%
% A master's thesis at the Faculty of Informatics has to be finished within six months. During this period regular meetings between the advisor(s) and the author have to take place.
% In addition, the following milestones have to be fulfilled:
% \begin{enumerate}
%   \item  Within one month after having fixed the topic of the thesis the master's thesis proposal has to be prepared and must be accepted by the advisor(s). The master's thesis proposal must follow the respective template of the dean of academic affairs. Thereafter the proposal has to be applied for at the deanery. The necessary forms may be found on the web site of the Faculty of Informatics. \url{http://www.informatik.tuwien.ac.at/dekanat/formulare.html}
%   \item  Accompanied with the master's thesis proposal, the structure of the thesis in terms of a table of contents has to be provided.
%   \item Then, the first talk has to be given at the so-called ``Seminar for Master Students''. The slides have to be discussed with the advisor(s) one week in advance. Attendance of the ``Seminar for Master Students'' is compulsory and offers the opportunity to discuss arising problems among other master students.
%   \item At the latest five months after the beginning, a provisional final version of the thesis has to be handed over to the advisor(s).
%   \item As soon as the provisional final version exists, a first poster draft has to be made. The making of a poster is a compulsory part of the ``Seminar for Master Students'' for all master studies at the Faculty of Informatics. Drafts and design guidelines can be found at \url{http://www.informatik.tuwien.ac.at/studium/richtlinien}.
%   \item After having consulted the advisor(s) the second talk has to be held at the ``Seminar for Master Students''.
%   \item At the latest six months after the beginning, the corrected version of the master's thesis and the poster have to be handed over to the advisor(s).
%   \item After completion the master's thesis has to be presented at the ``epilog''. For detailed information on the epilog see: \\ \url{http://www.informatik.tuwien.ac.at/studium/epilog}
% \end{enumerate}
%
% \section{Structure of the Master's Thesis}
%
% If the curriculum regulates the language of the master's thesis to be English (like for ``Business Informatics''), the thesis has to be written in English. Otherwise, the master's thesis may be written in English or in German. The structure of the thesis is predetermined.
% The table of contents is followed by the introduction and the main part, which can vary according to the content. The master's thesis ends with the bibliography (compulsory) and the appendix (optional).
%
% \begin{itemize}
%   \item	Cover page
%   \item Acknowledgements
%   \item Abstract of the thesis in English and German
%   \item Table of contents
%   \item Introduction
%   	\begin{itemize}
%   		\item motivation
%   		\item problem statement (which problem should be solved?)
%   		\item aim of the work
%   		\item methodological approach
%   		\item structure of the work
%   	\end{itemize}
%   \item State of the art / analysis of existing approaches
%   	\begin{itemize}
%   		\item literature studies
%   		\item analysis
%   		\item comparison and summary of existing approaches
%   	\end{itemize}
%   \item Methodology
%   	\begin{itemize}
%   		\item used concepts
%   		\item methods and/or models
%   		\item languages
%   		\item design methods
%   		\item data models
%   		\item analysis methods
%   		\item formalisms
%   	\end{itemize}
%   \item Suggested solution/implementation
%   \item Critical reflection
%   	\begin{itemize}
%   		\item comparison with related work
%   		\item discussion of open issues
%   	\end{itemize}
%   \item Summary and future work
%   \item Appendix: source code, data models, \dots
%   \item Bibliography
% \end{itemize}
%
